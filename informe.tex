\documentclass{article}
\usepackage[utf8]{inputenc}
\usepackage[letterpaper, margin=1in]{geometry}

\title{Miniproyecto FADA - Universidad del Valle}
\date{}
\author{David Santiago Cortés, Alejandro Orozco, Brayan Rincones}

\begin{document}
	\maketitle

	\section{Soluciones Planteadas}
		\textbf{Idea General de la solución:}\\
		Almacenar los animales y sus grandezas en una estructura de datos que almacene tuplas llave:valor, de la forma:
		\texttt{[...(animal:grandeza)...]}, se ordena este arreglo/lista de acuerdo a los valores de cada tupla allí
		almacenada y se utiliza para armar las escenas de todas las partes del evento.\\
		
		Para calcular la apertura se realiza un primer ciclo for que se ejecutará $(m-1)*k$ veces, en cada ciclo se calcula
		una escena para la apertura así:
		
		\subsection{$O(n^2)$}
			\textbf{Idea de la solución:}\\
			\textbf{Estructuras de datos utilizadas:} \\
			\textbf{Lenguaje en el que se implementó:}
		
		\subsection{$O(n*\log(n))$}
			\textbf{Idea de la solución:}\\
			\textbf{Estructuras de datos utilizadas:} \\
			\textbf{Lenguaje en el que se implementó:}
		\subsection{$O(n)$}
			\textbf{Idea de la solución:}\\	
			\textbf{Estructuras de datos utilizadas:} \\
			\textbf{Lenguaje en el que se implementó:}
	\section{Análisis de Resultados}

	\section{Instrucciones para la ejecución}

	\section{Sets de prueba}

	\section{Conclusiones del proyecto}

\end{document}

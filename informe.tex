\documentclass{article}
\usepackage[utf8]{inputenc}
\usepackage[letterpaper, margin=1in]{geometry}

\title{Miniproyecto FADA - Universidad del Valle}
\date{}
\author{David Santiago Cortés, Alejandro Orozco, Brayan Rincones}

\begin{document}
	\maketitle

	\section{Soluciones Planteadas}
		\textbf{Idea General de la solución:}\\
		Almacenar los animales y sus grandezas en una estructura de datos tipo lista, arreglo o diccionario, 
		se ordena esta estructura de acuerdo a los valores de las grandezas y se guarda utiliza para armar las
		escenas de todas las partes del evento.\\	
		\subsection{$O(n^2)$}
			\textbf{Idea de la solución:}\\
			\textbf{Estructuras de datos utilizadas:} \\
			\textbf{Lenguaje en el que se implementó:}
		
		\subsection{$O(n*\log(n))$}
			\textbf{Idea de la solución:}\\
			\textbf{Estructuras de datos utilizadas:} \\
			\textbf{Lenguaje en el que se implementó:}
		\subsection{$O(n)$}
			\textbf{Idea de la solución:} Después de recolectar los datos de entrada se construye una lista de tuplas a partir de la
			lista de animales y grandezas, luego se ordena ascendentemente de acuerdo a los valores de cada tupla utilizando Counting Sort.
			\textbf{Estructuras de datos utilizadas:} Tuplas y listas.\\
			\textbf{Lenguaje en el que se implementó:} Python
	\section{Análisis de Resultados}
		\subsection{$O(n^2)$}
		\subsection{$O(n*\log(n))$}
		\subsection{$O(n)$}
	\section{Instrucciones para la ejecución}

	\section{Sets de prueba}

	\section{Conclusiones del proyecto}

\end{document}
